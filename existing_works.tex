\section{Existing Works}

We review existing network intrusion detection systems that adopts
both classic machine learning and novel deep learning approaches.

\subsection{Classic Machine Learning Approaches}
Prior researchers modeled the intrusion detection task as an unsupervised
anomaly detection problem, and proposed a series of approaches.
Examples include Mahalanobis-distance based outliner detection~\cite{ComparativeAnomalyNIDS},
density-based outliner detection~\cite{LOF, ComparativeAnomalyNIDS},
evidence accumulation for ranking outliner~\cite{RankingOutliner}, etc.
One of the advantage of these unsupervised approaches is to tackle the problem of
the unavailability of labeled traffic data.

Alternatively, prior researchers made a lot of effort to obtain meaningful
attacking data and to convert them into large amount of labeled data~\cite{DARPA, KDDCup, NSL-KDD}.
Such efforts make it possible to apply supervised machine learning algorithms to the
intrusion detection problem.
Successfully applied approaches include decision trees~\cite{DecisionTree},
linear and non-linear support vector machines~\cite{SVM}, NB-Tree~\cite{NB-Tree} and so on.

\subsection{Deep Learning Flavor Approaches}
There are some pioneer works that introduced deep learning approaches to intrusion detection.
For example, \cite{STL-NIDS} adopts sparse autoencoder and the self-taught learning
scheme~\cite{SparseAE} to handle the problem that there is limited amount of labeled data
available for training supervised model.
Similar semi-supervised approach have also been applied to
Discriminative restricted Boltzmann machine~\cite{AnomalyDetectionRBM}.
