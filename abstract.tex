\section{Abstract}
Network intrusion detection system (NIDS) is essential to the security of any systems with communication capacity.
Recently a handful of novel deep neural networks bundled with more advanced and smarter
training algorithms have achieved unprecedentedly good performance on image classification,
natural language processing, speech recognition and many other research branches.
Motivated by these impressive improvements in the field of artificial intelligence,
this paper tries to answer the following questions:
Can we transfer hall-of-fame deep learning approaches to network intrusion detection task?
If yes, how much improvement can be expected?
If no, what are the reasons?

We answer these questions in four steps.
Firstly, we introduce deep learning models and techniques and why they may better solve the
network intrusion detection problem.
%Besides, we also survey and discuss the status quo of available network intrusion detection datasets
%and its implication on applying deep learning models to network intrusion detection.
Then we briefly review the existing machine learning solutions to network intrusion detection,
some of which provide the state-of-the-art detection performance.
After that, we describe several groups of the cutting-edge deep learning models
in concisely mathematical languages.
We conduct a quantitatively comparative study of each of them with two off-line network intrusion detection datasets,
with the help of our own TensorFlow-based deep learning library, NetLeaner.
Apart from making NetLearner publicly available, we also share the hacks and tricks
used during the training phase so that future researchers can easily reproduce and extend our work.
%[To the best of our knowledge, our feed-forward neural network achieves the best 5-classification
%result, with accuracy of 81.42\% and F1-Score of 80.44\%.]
