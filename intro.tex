\section{Introduction}

Network intrusion detection system (NIDS) is the essential security technology that
aims to protect a computer network intelligently and automatically.
As either a hardware device or software application,
it monitors a network for malicious activities or policy violations.
By intercepting and analyzing the bi-direction traffics through the network,
it raises alarm if intrusion, attack or violation are observed.
There are two general approaches to detect intrusions.
In signature based intrusion detection, e.g. SNORT~\cite{Snort},
rules for specific attacks are pre-installed in the system.
It report suspicious traffic when the traffic matches any signature of known attacks.
The major drawback of signature matching approach is that
it is only effective for previously detected attacks that have an identifiable signature.
As a result, signature database needs to be manually updated whenever a new type of attack
is discovered, with significant effort, by the network administrator.
Anomaly detection based approach overcomes these limitations by adopting a certain
type of machine learning technique to model the trustworthy network activities.
Traffics that significantly deviates from the built model are treated as malicious.
This idea have been shown to be able to detect unknown or novel attacks~\cite{NSL-KDD, STL-NIDS}.
However, if the built model for normal traffics are not generalized enough,
anomaly based approach will treat unforeseen normal traffic as malicious,
suffering from high false positive.

In this project, we follow the anomaly detection based idea, and tries to enhance it with the
state-of-art machine learning technology, e.g. various deep learning architectures.
Specifically, we have made the following contributions in this paper.

Firstly, we introduce the background of deep learning models and techniques and discuss why they may better solve the
network intrusion detection problem.
%Besides, we also survey and discuss the status quo of available network intrusion detection datasets
%and its implication on applying deep learning models to network intrusion detection.
Then we briefly review the existing machine learning solutions to network intrusion detection,
some of which provide the state-of-the-art detection performance.
After that, we describe several groups of the cutting-edge deep learning models
in concisely mathematical languages.
We conduct a quantitatively comparative study of each of them with two off-line network intrusion detection datasets~\cite{NSL-KDD, UNSW},
with the help of our own TensorFlow-based deep learning library, NetLeaner.
The detection performance is measured in accuracy, precision, recall and F-Score,
with detailed confusion matrix.

Apart from making NetLearner publicly available, we also share the hacks and tricks
used during the training phase so that future researchers can easily reproduce and extend our work.

Several deep learning approaches are first investigated and discussed,
including multi-layer perceptrons, restricted Boltzmann machine, generative adversarial nets and wide \& deep model.
We then present NetLearner, an implementation of all of the investigated approaches,
on the basis of TensorFlow.
We not only make the codebase of NetLearner publicly available to research community,
but also share the deep learning related hacks and tricks used during the training phase,
so that future researches can easily reproduce and extend our work.
