\section{Deep Learning Background}
We identify three main reasons why deep learning succeed in many areas that related
to artificial intelligence, and their implications on network intrusion detection problem.

\subsection{Learning/Training Techniques}
Given the feature representations and neural network model, learning is in essential an optimization
problem for minimizing a predefined cost function.
The most common used optimizing algorithm is back-propagation and gradient descent because it is Hessian-free.
The problem is that usually the cost function is non-convex with a lot of local minima.
Exploding and vanishing gradient will make back-propagation difficult to train deep models, such as recurrent neural networks.
Even if we can tolerate the long training time and carefully deal with gradient exploding and vanishing,
the trained model is usually overfitted to the training dataset, and not able to generalize well to the testing dataset or for future usage.
The emergence of novel learning algorithms and training techniques make training large and deep
neural network possible.
For example, stochastic gradient descent (SGD) with mini-batches can greatly increase the training speed.
In each step of gradient descent, it is shown that momentum can prevent SGD from oscillating across but pushing along the shallow ravine.
Decaying learning rate usually help us find better local minima.
To prevent overfitting, researchers have proposed dropout to average exponential number of neural network models.
These learning algorithms and training techniques will directly help neural networks achieve
better performance for the network intrusion detection problem since they are general to any types of neural models.


\subsection{Unsupervised Generative Models}
Another 

In the area of network security, the amount of network traffic data is enormously large,
usually in the order of terabytes per day in a large monitored network.
Such available big data makes deep learning techniques a promisingly better solution
to traffic classification.
In practice, however, the amount of data is impossible for a human security analyst or
a group of them to review, e.g., to find patterns and label anomalies.
Generative model which can be trained unsupervised comes to rescue in that
\begin{itemize}
    \item It utilizes the large amount of unlabeled data to learning useful and hierarchical features
        from the data itself;
    \item It is equivalently a way to initialize the weights of the hidden layers
        in a deep neural network, which can be further fine-tuned to be a high performance classifier.
\end{itemize}
In this project we propose to try two generative models: restricted Boltzmann machine and autoencoders.

\subsection{Datasets}

\begin{table*}[]
\centering
\caption{Popular Datasets used in Deep Learning v.s. Available Network Traffic Datasets}
\label{Tab:Datasets}
\begin{tabular}{c|l|l|l}
\multicolumn{1}{c|}{Domain}                           & Dataset Name  & \#Examples in Training Set & Feature Dimension                         \\
\hline
\hline
\multirow{6}{*}{Image}                               & MNIST         & 60,000                     & 784 (28$\times$28 gray images)                 \\
                                                     & SVHN          & 600,000                    & 3072 (32$\times$32 color images)                 \\
                                                     & CIFAR-10      & 60,000                     & 3072 (32$\times$32 color images)                 \\
                                                     & Flickr Photo  & 100,000,000                & O(100,000) pixels per image               \\
                                                     & Tiny          & 80,000,000                 & O(1000) pixels per image                  \\
                                                     & ImageNet      & 100,000,000                & O(16,384) (at least 128$\times$128 color images) \\
\hline
\multicolumn{1}{l|}{\multirow{2}{*}{Network Traffic}} & UNSW-NB15     & 175,341                    & 49                                        \\
\multicolumn{1}{l|}{}                                 & NSL-KDD       & 125,973                    & 41                                       
\end{tabular}
\end{table*}
